\chapter*{Resumen ejecutivo}
\section*{Introducción}

\tb{cuidado con usar dron y uav, pueden ser los militares... }
Los drones son vehículos no tripulados cuya popularidad en el mundo de la industria es cada vez mayor siendo utilizados para realizar tareas en campos muy diversos, como la inspección industrial, la industria cinematográfica o para su uso en operaciones de búsqueda y rescate. Dentro de este grupo los más empleados por la industria son los cuadricópteros, debido a la simplicidad de estos, lo que \tb{ reduce su peso y su coste}.

La inmensa mayoría del uso de estas aeronaves es mediante teleoperación, dentro del paradigma de los RPAS (\textit{Remotely Piloted Aircraft System}) aunque el auge de estas aeronaves han llevado a la comunidad científica hacia el desarrollo de sistemas que permita a los cuadricópteros ser capaces de realizar tareas de forma autónoma.

Una prueba de estos avances se pueden observar en diversas competiciones internacionales como IMAV, IARC, IROS Autonomous drone challenge o más recientemente el \tb{AIRR}. Competiciones en las que el objetivo es conseguir que UAV sean capaces de realizar pruebas complejas de forma autónoma.


%Aunque se ha conseguido automatizar ciertas tareas, como el análisis topográfico con UAV.

Dentro de estas competiciones, el Alphapilot 

% Hablar de  lo relevantes que son las carreras de drones





\section*{Alcance}

Los objetivos de este trabajo consisten en diseñar un controlador capaz de permitir que un cuadricóptero sea capaz de volar a muy altas velocidades, superiores a 5 m/s, a través de un circuito de carreras para drones, así como generar las trayectorias que el dron debe seguir para recorrer el circuito de forma óptima.

\section*{Solución realizada}

\begin{itemize}
	\item \textbf{Controlador pequeños ángulos:}
	\item \textbf{Controlador grandes ángulos:}
	\item \textbf{Generador de trayectorias óptimas en acceleracion:}
	\item \textbf{Generador de trayectorias óptimas en \textit{snap}:}
\end{itemize}

\section*{Experimentación y resultados.}

Para la simulación se ha empleado el entorno de simulación Flightgoogles \tb{citar}  el cual fue el empleado para las pruebas clasificatorias virtuales del Alphapilot 2019.



\section*{Conclusiones y trabajo futuro}

\section*{Palabras clave}
UAV, cuadricóptero, control clásico, aprendizaje automático, inteligencia artificial, aprendizaje por refuerzo, redes neuronales.

\section*{Códigos UNESCO}
\begin{itemize}
	\item[] $120304$ \quad INTELIGENCIA ARTIFICIAL
	\item[] $120326$ \quad SIMULACIÓN
	\item[] $330104$ \quad AERONAVES
	\item[] $330412$ \quad DISPOSITIVOS DE CONTROL
	\item[] $330703$ \quad DISEÑO DE CIRCUITOS 

\end{itemize}
\newpage
\cleardoublepage