\chapter*{Resumen ejecutivo}
\section*{Introducción}

Los drones de carreras autónomos aun están lejos de alcanzar el rendimiento de los pilotos humanos. \tb{añadir todos los campos que tienen relevancia: estimacion, control, generación de trayectorias, percepción y todo a muy altas velocidades con sistemas a bordo}

En 2019 la \textit{Drone Racing League} y la empresa \textit{Lockheed Martin} organizaron \textbf{Alphapilot} una carrera de drones autónomos con un gran premio de 1 millon de dólares para el ganador.

\section*{Alcance}

Los objetivos de este trabajo consisten en diseñar un controlador capaz de permitir que un cuadricóptero sea capaz de volar a muy altas velocidades, superiores a 5 m/s, a través de un circuito de carreras para drones, así como generar las trayectorias que el dron debe seguir para recorrer el circuito de forma óptima.

\section*{Solución realizada}

\begin{itemize}
	\item \textbf{Controlador pequeños ángulos:}
	\item \textbf{Controlador grandes ángulos:}
	\item \textbf{Generador de trayectorias óptimas en acceleracion:}
	\item \textbf{Generador de trayectorias óptimas en \textit{snap}:}
\end{itemize}

\section*{Experimentación y resultados.}

Para la simulación se ha empleado el entorno de simulación Flightgoogles \tb{citar}  el cual fue el empleado para las pruebas clasificatorias virtuales del Alphapilot 2019.



\section*{Conclusiones y trabajo futuro}

\section*{Palabras clave}
UAV, cuadricóptero, control clásico, aprendizaje automático, inteligencia artificial, aprendizaje por refuerzo, redes neuronales.

\section*{Códigos UNESCO}
\begin{itemize}
	\item[] $120304$ \quad INTELIGENCIA ARTIFICIAL
	\item[] $120326$ \quad SIMULACIÓN
	\item[] $330104$ \quad AERONAVES
	\item[] $330412$ \quad DISPOSITIVOS DE CONTROL
	\item[] $330703$ \quad DISEÑO DE CIRCUITOS 

\end{itemize}
\newpage
\cleardoublepage