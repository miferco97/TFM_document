%\documentclass[]{dinel-upm}
\documentclass[a4paper,twoside,12pt]{book}
%\setlength{\columnsep}{0.7cm} % space between columns
\usepackage{filecontents}
\usepackage{gensymb}

\usepackage{lipsum}
\usepackage{fancyhdr}
\usepackage{titlesec}
\usepackage[spanish,es-tabla]{babel}
\usepackage[utf8]{inputenc}
\usepackage{amsmath,amssymb,amsthm}

\usepackage{afterpage}
\usepackage[left=2.5cm,right=2.5cm,top=3cm,bottom=2.5cm]{geometry}
\usepackage{tocloft}
\usepackage{enumerate} 
\usepackage{tikz}
\usetikzlibrary{shapes,arrows}

\usepackage{wrapfig}
\usepackage{graphicx,tabularx}
\usepackage{eurosym}
\usepackage{multicol}
\usepackage{multirow}
\usepackage{rotating}
\usepackage{pdfpages}
\usepackage{pdflscape}
\usepackage{booktabs}
\usepackage[hyphens]{url}
\usepackage{ragged2e}
\usepackage{caption}
\DeclareCaptionListFormat{myfmt}{#1.#2}
\usepackage[list=true,listformat=myfmt]{subcaption}

%\usepackage{tikz}
\usepackage[cuteinductors,americanvoltages,americancurrents,american]{circuitikz}

\newcommand{\newItemUrl}[1]{\href{#1}{#1}}
\newcommand{\tabitem}{~~\llap{--}~~}
\newcommand{\newItem}[1]{\multicolumn{3}{|p{11cm}|}{\tabitem #1}\\}

\newcommand\Tstrut{\rule{0pt}{2.6ex}} 
\newcommand\Zstrut{\rule{0pt}{3.2ex}} 
\newcommand\Bstrut{\rule[-0.9ex]{0pt}{0pt}}
\newcommand{\HRule}{\rule{\textwidth}{0.1mm}} 		
\usepackage{hyperref}
					% horizontal line and its thickness

\titleformat
{\chapter} % command
[hang] % shape
{  \normalfont\bfseries\Huge} % format
{} % label
{0.5ex} % sep
{
%	\centering 
} % before-code
[] % after-code

%nuevo comando

\newcommand{\tb}[1]{\textcolor{blue}{#1}}
\usepackage[english]{cleveref} %con este paquete si usas \cref{label} te pone automáticamente figura,ecuacion..

\Crefname{figure}{Fig.}{Figs.}% {<type>}{<singular>}{<plural>}

\newcommand{\refAnexo}[1]{(Anexo \ref{#1})}
\newcommand{\refFigura}[1]{(Figura \ref{#1})}
\newcommand{\refApartado}[1]{(Apartado \ref{#1})}
\title{\textbf{Kit de desarrollo y validación de algoritmos de control de actitud para cuadricópteros}}

\author{Miguel Fernández Cortizas}
\date{}



\newcommand{\grad}{$^{\circ}$}
\usetikzlibrary{babel}
\usepackage{tikz}
\usetikzlibrary{matrix,chains,positioning,decorations.pathreplacing,arrows}

\usepackage{pgfplots}
\usepackage{ifthen}

\usepackage{amsmath}
\DeclareMathOperator{\sigm}{sigm}
\newcommand{\diff}{\mathop{}\!\mathrm{d}}

\usepackage{fancyhdr}

\fancyhf{}
\fancyhead[LE,RO]{\nouppercase{\rightmark}}
\fancyfoot[LE,RO]{\thepage}
%\fancyhead[LO,RE]{\nouppercase{\leftmark}}

\fancypagestyle{plain}{%
	\fancyhf{}% clears all header and footer fields
	\fancyfoot[LE,RO]{\thepage}%
	\renewcommand{\headrulewidth}{0pt}%
	\renewcommand{\footrulewidth}{0.0pt}
}

     

\newenvironment{dedication}
{%\clearpage           % we want a new page          %% I commented this
	\thispagestyle{empty}% no header and footer
	\vspace*{\stretch{1}}% some space at the top
	\itshape             % the text is in italics
	\raggedleft          % flush to the right margin
}
{\par % end the paragraph
	\vspace{\stretch{3}} % space at bottom is three times that at the top
	\clearpage           % finish off the page
}

\begin{document}
	
	\pagestyle{empty}
	\pagenumbering{gobble}
	
	\begin{landscape}
	\includepdf[page=-,angle=90]{portada/PORTADA_Trabajo_Fin_de_Grado.pdf}
	\end{landscape}
	\clearpage\null\newpage

	
\begin{center}
	\textbf{ UNIVERSIDAD POLITÉCNICA DE MADRID }\\
	ESCUELA TÉCNICA SUPERIOR DE INGENIEROS INDUSTRIALES\\
	\tb{GRADO EN INGENIERÍA EN TECNOLOGÍAS INDUSTRIALES}
	
	
	\vspace{1cm}
	
	\includegraphics[height = 10cm]{portada/logoupm}
	
	\vspace{1cm}
	
	{\LARGE \tb{KIT DE DESARROLLO Y VALIDACIÓN DE}}
	\\
	
	\vspace{0.2cm}
	{\LARGE	\tb{ALGORITMOS DE\ CONTROL DE ACTITUD}}
		 \vspace{0.2cm}
		 \tb{{\LARGE PARA CUADRICÓPTEROS.}}
	
	\vspace{2cm}
	
	{\Large Miguel Fernández Cortizas}\\
	
	
	
	
	\vspace{2cm}
	
	{\large  Tutor académico:}\\
	\vspace{0.2cm}
	{\Large D. Pascual Campoy Cervera }
	
	\vspace{0.5cm}
	
	
	\vfill{\Large Madrid - Espa\~{n}a\\
		2020}

\end{center}
\newpage


	\clearpage\null\newpage

\begin{dedication}
	En memoria de mi padrino Juan, \\
	seguiré trabajando hasta alcanzar las metas\\
	que me hubiese gustado celebrar contigo. 
\end{dedication}

	\justify
	\pagestyle{empty}
	
\chapter*{Agradecimientos}

En primer lugar, quiero darle las gracias a Carmen, por impulsarme a ser mejor persona día tras día con su cariño y su apoyo.

Asimismo, quiero darle las gracias a Pascual Campoy por ofrecerme la oportunidad de trabajar y aprender dentro del CVAR y por brindarme la oportunidad de conocer a muy buenos compañeros, los cuales me han motivado, con su pasión y esfuerzo, a continuar trabajando en mundo de la investigación.

Finalmente me gustaría dar las gracias a Carlos Redondo, compañero y amigo con el que he compartido grandes momentos estos años de universidad y con el que he trabajado mano a mano en incontables ocasiones, apoyándonos y motivándonos mutuamente. 





	\pagestyle{empty}			
	\chapter*{Resumen ejecutivo}
\section*{Introducción}

Los pequeños multirrotores no tripulados (comúnmente llamados drones) son vehículos cuya popularidad en el mundo de la industria es cada vez mayor, siendo utilizados para realizar tareas en campos muy diversos, como la inspección industrial, la industria cinematográfica o para su uso en operaciones de búsqueda y rescate. Aunque la inmensa mayoría del uso de estas aeronaves es mediante teleoperación, el auge de estas aeronaves han llevado a la comunidad robótica hacia el desarrollo de sistemas capaces de realizar tareas de forma autónoma.

Con el ánimo de impulsar el desarrollo de las tecnologías necesarias para conseguir mejorar la autonomía de estos drones, existen diversas competiciones internacionales como el IARC (International Aerial Robotics Competition), IMAV (International Micro Air Vehicle Competition) o MBZIRC (Mohamed Bin Zayed International Robotics Challenge), en las que el objetivo es conseguir que las aeronaves sean capaces de realizar pruebas complejas de forma autónoma. 

Dentro de estás competiciones internacionales, existen las carreras de drones autónomos, como ADR o AlphaPilot, en las que el objetivo es que un dron sea capaz de recorrer un circuito de forma autónoma en el menor tiempo posible.

\section*{Alcance}
Los objetivos de este trabajo consisten en diseñar un sistema autonómo, para que un cuadricóptero sea capaz de recorrer un circuito de carreras de forma satisfactoria y en diseñar un controlador capaz de conseguir que un cuadricóptero vuele a altas velocidades a través de un circuito, así como generar las trayectorias que el aeronave debe seguir para recorrer el circuito de forma óptima. Adicionalmente, los algoritmos están diseñados para ser ejecutados en un ordenador a bordo, por lo que se deben optimizar para obtener el mayor rendimiento.
El desarrollo y validación del trabajo se realizará empleando el simulador y las reglas empleadas en la prueba virtual clasificatoria del AlphaPilot2019.
\section*{Solución realizada}
\begin{itemize}
\item \textbf{Controlador:}
Para conseguir que el cuadricóptero recorra el circuito a altas velocidades se han implementado dos controladores, uno linealizado en torno al punto de equilibrio, en los que la orientación del cuadricóptero varía un pequeño ángulo respecto al estado de hover ($\phi,\theta  < \pi/6 $). y otro sin linealizar, para cuando esta variación es de un gran ángulo ($\phi,\theta  \ge \pi/6 $).

\item \textbf{Generación de trayectorias:} Se ha dividido la generación de trayectorias en dos partes:
Una trayectoria que abarca todo el recorrido del circuito y otra más corta dentro de un horizonte temporal próximo a la posición del cuadricóptero. Ambas son trayectorias óptimas, generadas por \textit{splines}.

\item \textbf{Arquitectura del sistema:} Los algoritmos de control y generación de trayectorias se han integrado dentro de una arquitectura modular empleando el \textit{framework} de robótica ROS. Adicionalmente se han desarrollado módulos de percepción y estimación de estado empleando datos provistos por el simulador.

\end{itemize}
	

\section*{Experimentación y resultados.}

Para realizar los experimentos se ha empleado el entorno de simulación Flightgoogles \cite{guerra2019flightgoggles}  el cual fue el empleado para las pruebas clasificatorias virtuales del Alphapilot 2019. Durante el transcurso del trabajo se han realizado experimentos para la evaluación de los controladores implementados, en los que se prueba el rendimiento de los mismos recorriendo trayectorias sencillas a distintas velocidades. Asimismo, se han realizado experimentos recorriendo el circuito de carreras entero, en los que se prueba el rendimiento del generador de trayectorias y de la arquitectura diseñada.

En estos experimentos se puede observar como el controlador para grandes ángulos es capaz de alcanzar velocidades más altas (hasta 10 m/s) con un menor error de seguimiento que el controlador linealizado. La arquitectura presentada es capaz de recorrer el circuito entero de forma satisfactoria en menos de 25 segundos, en su configuración más agresiva, tiempo que se encuentra dentro de los 3 mejores tiempos obtenidos por los equipos participantes en las clasificatorias del AlphaPilot2019 \cite{guerra2019flightgoggles} \footnote{Estos resultados fueron presentados en el Workshop  \textit{Perception and Control for Fast and Agile Super-Vehicles} de la conferencia RSS20, integrando los módulos de percepción y estimación de estados desarrollados por Carlos Redondo Plaza}. En el enlace \url{https://vimeo.com/465394661} se encuentra un vídeo del sistema recorriendo el circuito en su configuración más conservadora. 


\section*{Conclusiones y trabajo futuro}

Se ha conseguido desarrollar un sistema modular capaz de recorrer un circuito de carreras de forma autónoma satisfactoriamente. El controlador implementado permite realizar trayectorias a velocidades muy elevadas, aunque para conseguir disminuir el error de seguimiento sería posible emplear un controlador MPC que sea capaz de mitigar los cambios bruscos en las referencias. Finalmente, la decisión de dividir la generación de trayectorias en dos partes ha permitido computar ambas trayectorias a una alta frecuencia siendo capaces de adecuarse correctamente a los cambios en las estimaciones de las posiciones de las puertas en el circuito. En versiones futuras, sería conveniente tener en cuenta restricciones espaciales en la generación de las mismas.

La arquitectura propuesta permite su extensión con nuevos módulos de forma sencilla, por lo que se podría emplear para realizar otras tareas mediante la integración de los módulos necesarios.

\newpage
\section*{Palabras clave}
UAV, cuadricóptero, modelado dinámico, teoría de control, generación de trayectorias, sistemas autónomos.

\section*{Códigos UNESCO}
\begin{itemize}
	\item[] $120326$ \quad SIMULACIÓN
	\item[] $330104$ \quad AERONAVES
	\item[] $330412$ \quad DISPOSITIVOS DE CONTROL
	\item[]	$330118$ \quad ESTABILIDAD Y CONTROL
	\item[] $330417$ \quad SISTEMAS EN TIEMPO REAL 
		

\end{itemize}
\newpage
\cleardoublepage
	
	\tableofcontents
%	\cleardoublepage\null\newpage
	
	\pagestyle{fancy}
	\justify
	\pagenumbering{arabic}
	\setcounter{page}{0}
	\thispagestyle{empty}
	
	\chapter{Introducción}
\tb{cuidado con usar dron y uav, pueden ser los militares... }
Los drones son vehículos no tripulados cuya popularidad en el mundo de la industria es cada vez mayor siendo utilizados para realizar tareas en campos muy diversos, como la inspección industrial, la industria cinematográfica o para su uso en operaciones de búsqueda y rescate. Dentro de este grupo los más empleados por la industria son los cuadricópteros, debido a la simplicidad de estos, lo que \tb{ reduce su peso y su coste}.

La inmensa mayoría del uso de estas aeronaves es mediante teleoperación, dentro del paradigma de los RPAS (\textit{Remotely Piloted Aircraft System}) aunque el auge de estas aeronaves han llevado a la comunidad científica hacia el desarrollo de sistemas que permita a los cuadricópteros ser capaces de realizar tareas de forma autónoma.

Una prueba de estos avances se pueden observar en diversas competiciones internacionales como IMAV, MBZIRC o más recientemente el \tb{AIRR}. Competiciones en las que el objetivo es conseguir que UAV sean capaces de realizar pruebas complejas de forma autónoma.


%Aunque se ha conseguido automatizar ciertas tareas, como el análisis topográfico con UAV.

Dentro de estas competiciones, el Alphapilot 

% Hablar de  lo relevantes que son las carreras de drones

\section{Motivación}

Las velocidades a las que suelen volar estos uav normalmente no suelen exceder los 3 m/s estando la inmensa mayoría de ellas por debajo del 1.5 m/s. El aumentar la velocidad de vuelo, exige tener algoritmos de percepción y de estimación de estado más precisos y rápidos, así como algoritmos de planificación y de control más rápidos y ligeros.

Además en la competición toda la computación se realiza a bordo de la aeronave en un SBC, por lo que la capacidad de cálculo es limitada.

Generar un sistema autonónomo desde cero presenta una gran complejidad lo que escaparía del alcance de este trabajo, es por eso que en este trabajo se han simplificado el desarrollo de los módulos de estimación de estado y de percepción del circuito, empleando datos provistos por el simulador que permiten que el trabajo se haya centrado en el módulo de control y de generación de trayectorias.

\section{Solución propuesta}


En este trabajo se propone un sistema autónomo capaz de recorrer un circuito de carreras 
simulado a altas velocidades de forma autónoma y con incertidumbre sobre el recorrido en sí.

Para desarrollar los algoritmos se empleará el simulador fotorrealista FlightGoogles, empleado para las pruebas clasificatorias del AIRR 2019 como entorno de pruebas.



%Este sistema estará formado por 3 partes principales:
%\begin{itemize}
%	\item \textbf{Estimación de estado:} Para poder recorrer el circuito es necesario tener una estimación precisa del estado de la aeronave a lo largo del tiempo. Para poder localizar el desarrollo de los 2 bloques posteriores se empleará la estimación provista por el simulador.
%	\item \textbf{}
%	\item \textbf{Generación de trayectorias:}
%\end{itemize}



 
\section{Objetivos}

	\chapter{Estado del arte}

Para contextualizar el trabajo desarrollado dentro del campo de las carreras de drones autónomos es necesario conocer los avances obtenidos por la comunidad robótica. Debido a que los contenidos del trabajo se han enfocado, principalmente, en torno al controlador y a la generación de trayectorias se ha profundizado en el estado del arte de estos campos enfocados a los drones de carreras. Adicionalmente, debido al objetivo del trabajo de generar un sistema coordinado, también se ha revisado el trabajo realizado por los finalistas de las carreras de drones autónomos más importantes.

\section{Control y generación de trayectorias}

En la primera década de los 2000, la mayoría del trabajo realizado con multirrotores empleaba controladores linealizados en torno al punto de equilibrio (\textit{hover}), los cuales, unicamente garantizan la estabilidad de la aeronave para pequeños ángulos de \textit{pitch} y \textit{roll} \cite{hoffmann2008quadrotor}. En cuanto a las trayectorias generadas, la mayoría de ellas son trayectorias polinómicas del tipo \textit{spline}, generadas interpolando una función entorno a los puntos de paso deseados \cite{vanek2005}\cite{barrientos2009}.

En 2008 V. Raffo et al. \cite{MPCRaffo2008} emplearon una estructura de control basada en un controlador predictivo basado en el modelo (MPC) que se encargaba de seguir la trayectoria y un controlador $\mathcal{H}_\infty$ que controlaba la rotación de la aeronave. Con esta estructura son capaces de seguir trayectorias sencillas de forma robusta ante perturbaciones.

En 2010, Guillula et al. \cite{gillula2010design} diseñaron un controlador capaz de realizar maniobras acrobáticas, como una voltereta hacia atrás, con un cuadricóptero de forma segura. Para ello emplearon un \textit{framework} para el diseño de regiones de cambio seguras, en las que cada región presenta un modelo dinámico distinto. Sin embargo, estas maniobras se generan de forma discontinua, necesitando analizar cada parte de la trayectoria de forma independiente y generar situaciones de cambio entre estos modos de forma segura. 

En 2011, Mellinger et al. \cite{MinimunSnap2011} presentan un controlador para cuadricópteros que permite realizar maniobras agresivas en un espacio tridimensional de forma continua. Este controlador no está linealizado en torno a ningún punto de funcionamiento, por lo que permite seguir trayectorias agresivas con un bajo error de seguimiento aunque el aeronave tenga ángulos grandes de \textit{roll} y \textit{pitch}. Este es uno de los controladores más usados actualmente debido al rendimiento que consigue con un algoritmo sencillo y con un bajo coste computacional.

En 2012, Mallikarjunan et al. \cite{mallikarjunan2012l1} diseñaron un controlador de actitud adaptativo, aplicando control $\mathcal{L}_1$, capaz de seguir trayectorias de forma precisa y robusta, con presencia de incertidumbres en el modelo de la aeronave y de las perturbaciones del entorno.

En 2016, Kamel et al. \cite{KamelMPC2016} comparan el rendimiento de dos MPCs, uno lineal y uno no lineal, en el seguimiento de trayectorias agresivas con un cuadricóptero. En estos experimentos observaron que, aunque ambos controladores eran capaces de seguir las trayectorias de forma satisfactoria, el controlador no lineal, conseguía un rendimiento ligeramente superior.

En 2017, Faessler et al. \cite{Faessler17ral} emplearon control LQR considerando tanto la dinámica del cuadricóptero, como la dinámica aislada de cada rotor. Además, consideran los limites de los rotores para priorizar la saturación de aquellas entradas que son relevantes para la estabilización del cuadricóptero. Asimismo, en 2018 \cite{Faessler18ral}, refinaron el controllador de Mellinger et al. considerando el arrastre (\textit{drag}) de los rotores dentro del modelo dinámico del cuadricóptero, en lugar de considerarlo como una perturbación externa desconocida, consiguiendo una ligera mejora en el seguimiento de trayectorias a alta velocidad.
 
En 2018, Falanga et al. \cite{falanga2018pampc} presentan un controlador MPC consciente de la percepción, el cual unifica el control y la planificación para satisfacer objetivos de acción y percepción de forma simultánea. Las trayectorias generadas por el MPC deben tener en cuenta ambos objetivos, para conseguir realizar maniobras complicadas mientras maximizan la visibilidad de puntos de interés por la aeronave.


\section{Carreras de drones autónomos}
Ganaron la competición IROS 2018 Autonomous Drone Race \cite{BeautyAndTheBeast}.

Recientemente \cite{foehn2020alphapilot}





	\input{capitulos/background}
	\input{capitulos/hardware}
	\input{capitulos/software}
	\chapter{Metodología}
	
En los capítulos anteriores se ha presentado los distintos algoritmos de control y generación de trayectorias que se emplearán para intentar recorrer un circuito con un cuadricóptero de carreras autónomo en un entorno simulado. En este apartado se presenta la metodología empleada para superar el circuito que fue propuesto durante las pruebas clasificatorias del Alphapilot 2019.

\section{Sistemas de referencia}
Antes de continuar con la metodología empleada, es conveniente fijar los sistemas de referencia que se van a emplear. Primeramente se empleara un sistema de coordenadas global en el mundo. El origen de este sistema de referencia global fijo viene dado por el simulador y se encuentra en el centro del circuito. A partir de este sistema estableceremos el sistema de referencias móvil del aeronave, cuyo origen se encuentra en el centro de masas del cuadricóptero y las direcciones de los ejes siguen el estándar ENU (\textit{East North Up}). Finalmente, se generará un sistema de referencia móvil para cada puerta del circuito, cuyos orígenes corresponderán al centro de las puertas. 

\begin{figure}[htb!]
	\centering
	\includegraphics[width=0.75\textwidth]{imagenes/frames}
	\caption{Transformaciones entre los sistemas de referencia de las puertas, el cuerpo del aeronave y el mundo.}
	\label{waypoints:Refs}
\end{figure}



\section{Generación de \textit{waypoints}}
\label{section:gen_traj}
Para recorrer el circuito de forma satisfactoria es necesario que del aeronave atraviese las distintas puertas o \textit{gates} que componen el circuito en un orden concreto. Para conseguir esto es necesario conocer las posiciones de las puertas en el mundo y generar los puntos de paso necesarios para que del aeronave pase a través de ellas sin colisionar.

\begin{figure}[htb!]
	\centering
	\includegraphics[width=\textwidth]{imagenes/diagramacircuito}
	\caption{Vista aérea del circuito en el simulador FightGoggles, las puertas que se deben traspasar se simbolizan con su número en color amarillo.}
	\label{waypoints:circuito}
\end{figure}

Como se puede observar en la figura \ref{waypoints:circuito} del aeronave debe recorrer 11 puertas, cada una con un número de identificación, en un orden concreto. Cada una de estas puertas posee unas medidas estandarizadas de 2x2 m.

En la competición se proporciona el orden en el que se deben atravesar las puertas y una posición aproximada de las posiciones de cada una de ellas en el mundo. Esta posición aproximada posee un error significativo, por lo que es necesario corregir la estimación de la posición de estas puertas a medida que del aeronave avanza por el circuito (estimación \textit{online}).
\begin{figure}[htb!]
	\centering
	\includegraphics[width=0.68\textwidth]{imagenes/red_points}
	\caption{La imagen provista por el simulador con las esquinas de las puertas visibles marcadas en rojo.}
	\label{redpoints}
\end{figure}


Para poder realizar la estimación \textit{online} de las puertas, es necesario percibirlas, para ello se hace uso de las cámaras integradas en del aeronave simulada. Las cámaras del cuadricóptero permiten localizar las distintas puertas a lo largo del circuito. La imagen provista por el simulador, además de contener la imagen observada por la cámara, contiene también las posiciones en el plano imagen, de las esquinas de las puertas visibles (Figura \ref{redpoints}), así como el identificador de puerta a la que corresponde.



Esta información extra provista por el simulador facilita enormemente la estimación de las posiciones de las distintas puertas, reduciendo también el tiempo de calculo requerido.



Dado que se cuenta con los parámetros físicos de la cámara y con las dimensiones de las puertas, es posible emplear un algoritmo PnP (\textit{Perspective n-Points}) para calcular la posición relativa de los \textit{frames} con respecto a del aeronave. Concretamente, se ha empleado la medida de la posición del centro de la puerta, ya que nos permite generar los waypoints de forma más sencilla.






La incertidumbre de estas medidas disminuye a medida que el aeronave se acerca a las puertas, por lo que cuando las imágenes se toman a una distancia lejana, el error que tienen es elevado. Para disminuir la influencia de las medidas erróneas cada medida se somete a un filtrado en dos pasos:

\begin{enumerate}
	\item \textbf{Región de confianza:} Dado que siempre se posee una posición estimada de cada puerta es posible emplear esta información para descartar medidas erróneas. Siendo $\hat{G_i}$ la estimación previa del centro de la puerta $i$, se establece una bola $B(\hat{G_i},r)$ con centro en $\hat{G_i}$  y radio $r$ como región de confianza, es decir, si una nueva medida $G_i \notin B(\hat{G_i},r)$ entonces se desecha como una medida errónea.
	El valor del radio $r$ influye en la distancia máxima que puede tener una medida respecto a la estimación original para considerarse correcta. Dado que al comienzo del circuito se tienen unas posiciones de las puertas con un error muy grande, este valor $r$ no puede ser muy bajo, si no no se conseguiría corregir la posición de las puertas con mediciones correctas. Por el contrario, si el valor de $r$ es muy grande, este filtrado carecería de sentido, ya que cualquier medición se consideraría válida. En la practica se han probado con distintos valores de este parámetro, obteniendo mejores resultados con valores de $r$ de entre 4 y 6 metros.
	\item \textbf{Media móvil:} Si la medida obtenida $G_i$ se encuentra contenida en esa región, entonces, en lugar de sustituir directamente la estimación de la posición de esta puerta, se realiza una modificación de la medición anterior, de acorde a la fórmula:
	\begin{equation}
		\hat{G_i} = \alpha \hat{G}_{i,prev} + (1-\alpha)G_i
	\end{equation}
	siendo $\hat{G}_{i,prev}$ la estimación anterior y $\alpha \in (0,1)$ un parámetro de filtrado. Cuando $\alpha$ tiene valores pequeños, la estimación cambia rápidamente con las nuevas medidas, mientras que si $\alpha$ posee valores altos, la estimación varía ligeramente con cada una de las nuevas medidas. Los valores que se suelen emplear son $\alpha = 0.9$ y $\alpha = 0.99$. Para este filtrado se ha empleado un valor de $\alpha = 0.9$ dado que presenta un buen equilibrio entre robustez y reactividad ante nuevas medidas.

\end{enumerate}


Con este proceso se consiguen actualizar las posiciones estimadas de las puertas con una frecuencia aproximada de 60 Hz, la frecuencia de refresco de las cámaras. En los casos en los que se encuentran varias puertas en la imagen se realizan el filtrado anterior para cada una de ellas por separado.

\section{Trayectorias a largo y corto plazo}

Con la posición estimada de cada puerta en el mundo es posible construir la trayectoria que el dron debe seguir para completar el circuito. El problema es que estas posiciones cambian continuamente, siendo más fiables cuanto más cerca se encuentre el aeronave de la puerta, por lo que es necesario actualizar estas trayectorias de la forma mas ágil posible. Para conseguir que el sistema sea capaz de reaccionar de forma fluida a los cambios en la estimación de las puertas, se ha dividido la generación de trayectorias en dos partes : una a largo plazo que tiene en cuenta las posiciones de las puertas y una a corto plazo que genera trayectorias óptimas en un horizonte temporal finito.

Dado que el circuito es amplio y no existen obstáculos entre dos puertas sucesivas, no se tienen en cuenta los obstáculos del circuito para la generación de trayectorias. Para asegurar esto, es necesario que las trayectorias generadas no se alejen demasiado de las lineas rectas que unen las distintas puertas sucesivas. A continuación, se explicará de forma más detallada cómo se generan estas trayectorias.
\subsection{Trayectoria a largo plazo}

Para poder generar trayectorias suaves a lo largo de todo el circuito es conveniente tener en cuenta las posiciones aproximadas de todas las puertas,  de esta forma a medida que el aeronave recorre la trayectoria obtiene nuevas estimaciones que permiten ir refinando la trayectoria. Si se generase la trayectoria solo teniendo en cuenta los $m$ siguientes waypoints, la trayectoria a seguir variaría mucho después de pasar una puerta, lo que perjudicaría en el rendimiento del aeronave. Dado que el dron obtiene nuevas mediciones a una frecuencia de unos 60 Hz esta trayectoria se tiene que ir actualizando continuamente, por lo que es conveniente minimizar el tiempo de cómputo de cada una. 

Como se ha presentado en el capítulo \ref{cap:gen_tray} las trayectorias que se van a emplear son \textit{splines} en las que el grado de los segmentos polinómicos depende de la acción de control a minimizar. Los dos factores que afectan principalmente al tiempo de cómputo son el grado de los segmentos polinomiales, ya que a mayor grado, mayor número de coeficientes se deben calcular, y el número de \textit{waypoints}, ya que aumenta el número de segmentos polinomiales.

Para poder generar una ruta a través de todo el circuito que se calcule de la forma más rápida posible y que tenga en cuenta las limitaciones físicas del aeronave, se ha empleado trayectorias de aceleración mínima ($n = 2$) con 12 \textit{waypoints}, 11 de las puertas y 1 de la posición del aeronave. Estas trayectorias son rápidas de calcular y permiten actualizar la trayectoria con una frecuencia superior a la de adquisición de la cámara. Estas trayectorias cambian continuamente, por lo que no es conveniente usar estas trayectorias para controlar el cuadricóptero ya que las referencias del controlador estarían cambiando continuamente, generando un movimiento muy poco fluido. Para solucionar esto se emplean las trayectorias a corto plazo.

\begin{figure}[htb!]
	\centering
	\includegraphics[width=1\textwidth]{imagenes/Rviz_traj_largo}
	\caption{Visualización de las trayectoria a largo (blanco) y a corto (verde) plazo en Rviz.}
	\label{rviz:traj_largo}
\end{figure}
\newpage

\subsection{Trayectoria a corto plazo}

Para generar la trayectoria que va a seguir el controlador, es conveniente generar trayectorias de \textit{snap} mínimo ($n=4$), ya que son las que generan trayectos más suaves. El grado de los polinomios que componen estas \textit{splines} son de grado 7, lo que implica un elevado coste computacional. Además, es conveniente que los \textit{waypoints} empleados para generar esta trayectoria estén cerca (una distancia inferior a los 8 m entre waypoints), ya que si se eligen waypoints muy separados, la trayectoria resultante puede separarse demasiado de la linea recta entre ellos, llegando a poder colisionar con el entorno. Por lo que para generar una trayectoria completa del circuito sería necesaria una gran cantidad de waypoints, por lo que el tiempo de cómputo sería muy elevado, haciendo que el sistema no sea capaz de reaccionar a cambios en las posiciones de las puertas de forma rápida.

Para poder conseguir trayectorias de control óptimas se ha decidido partir de la trayectoria a largo plazo y establecer un horizonte temporal, sobre el cual se calcule la trayectoria a corto plazo. Este proceso se genera en varias etapas:
\begin{enumerate}
	\item Se localiza la posición del cuadricóptero dentro de la trayectoria a largo plazo. Para ello se parte de la última posición conocida de el aeronave en la trayectoria $t_c$ y se comprueba en un pequeño intervalo, en que posición se encuentra el aeronave. Dado que el sistema tiene inercia y los algoritmos tardan un tiempo en procesarse, es necesario recalcular esta posición para obtener una mayor precisión.
	
	\item Se muestrea la trayectoria a largo plazo para obtener los waypoints de la trayectoria a corto plazo. Partiendo de la posición obtenida previamente $t_c$, se fija un horizonte temporal en el que se quiere calcular la trayectoria a corto plazo $t_h$, también se fija la distancia a la que se quiere que se encuentren los waypoints $t_d$. Con estos datos se  muestrea la trayectoria en el intervalo $t \in (t_c, t_c + t_h)$ con una distancia de muestreo $t_d$ entre muestras. Las muestras obtenidas constituyen el conjunto de \textit{waypoints} que se emplearán para generar la trayectoria a corto plazo.
	
	\item Finalmente, se genera una trayectoria de \textit{snap} mínimo empleando el conjunto de waypoints obtenidos previamente. La trayectoria obtenida $P(t) :[0,t_f] \rightarrow \mathbb{R}^3$ proporciona las posiciones en el espacio tridimensional en las que deberá estar el cuadricóptero para cada instante de tiempo. Para generar las consignas del controlador es conveniente conocer también las velocidades y acceleraciones para cada instante de tiempo, por lo que se derivan estos splines para obtener la trayectoria en velocidad $V(t) :[0,t_f] \rightarrow \mathbb{R}^3$ y la trayectoria en acceleración  $A(t) :[0,t_f] \rightarrow \mathbb{R}^3$ siendo $t_f$ el tiempo de finalización de la trayectoria.
\end{enumerate}

Para conseguir que estas trayectorias tengan en cuenta los cambios $P(t)$, $V(t)$ y $A(t)$ en la estimación de las puertas, estas trayectorias se recalculan periódicamente a medida que se modifica la trayectoria a largo plazo.

\subsection{Orientación del aeronave en \textit{yaw}}
Además de las referencias de posición en el espacio tridimensional, para controlar el estado del aeronave es necesario indicar el angulo de \textit{yaw} deseado. De cara a obtener las mejores estimaciones en las posiciones de las puertas, se desea que el aeronave siempre este orientada de forma que las camáras miren hacia delante. Partiendo de la trayectoria a corto plazo $V(t)$ es facil obtener el valor requerido del angulo de \textit{yaw} $\psi$. Siendo $v_x(t)$ y $v_y(t)$ las componentes de la trayectoria $V(t)$ para el eje x y el eje y respectivamente, en ángulo de yaw deseado es:

\begin{equation}
	\psi_{des}(t) = -atan2(v_x,v_y) + \pi/2
\end{equation}
siendo $atan2(x,y)$ el arco tangente de dos parámetros entre $x$ e $y$. Con esta simple expresión se consigue que el aeronave se oriente de forma que las cámaras miren hacia donde se va a mover el aeronave.



















	\chapter{Experimentos}

\section{Controlador pequeños ángulos}
\begin{figure}[htb!]
	\centering
	\includegraphics[width=\textwidth]{imagenes/circle3ms}
	\includegraphics[width=\textwidth]{imagenes/circle5ms}
	\includegraphics[width=\textwidth]{imagenes/circle6ms}
	\caption{Trayectorias circulares a 3, 5 y 6 m/s empleando un controlador para angulos pequeños}
	\label{circle:slow}
\end{figure}
\section{Controlador grandes  ángulos}

\begin{figure}[htb!]
	\centering
	\includegraphics[width=\textwidth]{imagenes/fast_circle6ms}
	\includegraphics[width=\textwidth]{imagenes/fast_circle10ms}
	\includegraphics[width=\textwidth]{imagenes/fast_circle11ms}
	\caption{Trayectorias circulares a 6, 10 y 11 m/s empleando un controlador para angulos grandes}
	\label{circle:fast}
\end{figure}

\section{Circuito completo}
\subsection{\tb{Normal speed}}

\begin{figure}[htb!]
	\centering
	\includegraphics[width=\textwidth]{imagenes/circuitFigure}
	\caption{}
	\label{exp1:1}
\end{figure}

\begin{figure}[htb!]
	\centering
	\includegraphics[width=\textwidth]{imagenes/positionFigure}
	\caption{}
	\label{exp1:2}
\end{figure}

\begin{figure}[htb!]
	\centering
	\includegraphics[width=\textwidth]{imagenes/errorFigure}
	\caption{}
	\label{exp1:3}
\end{figure}



\begin{table}[htb!]
	\centering
	
	\begin{tabular}{l|c|c|c|}
		$\empty$&Eje x&Eje y&Eje z\\
		\midrule
		Error máximo (m)&1.5685&1.1577&0.7707\\
		Error medio (m) &0.3143&0.7306&0.0796\\
		
	\end{tabular}
	\caption{Errores de seguimiento durante el recorrido del circuito.}
\end{table}


\begin{table}[htb!]
	\centering
	\begin{tabular}{l|c|c|}
		$\empty$&Máxima& Media\\
		\midrule
		Velocidad (m/s)&7.4199&5.8565\\
		
	\end{tabular}
	\caption{Velocidad del cuadricóptero durante el recorrido del circuito.}
\end{table}


\subsection{\tb{BESTspeed}}

\begin{figure}[htb!]
	\centering
	\includegraphics[width=\textwidth]{imagenes/best_circuitFigure}
	\caption{}
	\label{exp1:1}
\end{figure}

\begin{figure}[htb!]
	\centering
	\includegraphics[width=\textwidth]{imagenes/best_positionFigure}
	\caption{}
	\label{exp1:2}
\end{figure}

\begin{figure}[htb!]
	\centering
	\includegraphics[width=\textwidth]{imagenes/best_errorFigure}
	\caption{}
	\label{exp1:3}
\end{figure}



\begin{table}[htb!]
	\centering
	\begin{tabular}{l|c|c|c|}
		$\empty$&Eje x&Eje y&Eje z\\
		\midrule
		Error máximo (m)&1.9776&1.9181&1.3302\\
		Error medio (m) &0.5973&1.4826&1.1413\\
		
	\end{tabular}
	\caption{Errores de seguimiento durante el recorrido del circuito.}
\end{table}



\begin{table}[htb!]
	\centering
	\begin{tabular}{l|c|c|}
		$\empty$&Máxima& Media\\
		\midrule
		Velocidad (m/s)&10.4948&9.3936\\
		
	\end{tabular}
	\caption{Velocidad del cuadricóptero durante el recorrido del circuito.}
\end{table}






\section{Experimentos en simulación}
\section{Experimentos en real}
	\input{capitulos/discusion}
	\chapter{Conclusiones y trabajo futuro}

\section{Conclusiones}

Durante el transcurso de este trabajo se ha desarrollado la arquitectura modular para un cuadricóptero de carreras autónomo. Esta modularidad de la arquitectura ha facilitado el desarrollo de los algoritmos de forma aislada así como el desarrollo de los experimentos de control, en los que se ha podido intercambiar los módulos de generación de trayectorias por módulos más sencillos de forma ágil. Esta modularidad también permitiría sustituir el módulo del simulador por un módulo de interfaz con una plataforma real, lo que facilitaría la realización de experimentos en real.

En cuanto al control del cuadricóptero se han conseguido implementar dos controladores del estado del arte de forma satisfactoria adaptándolos a las señales de control requeridas por el cuadricóptero. El empleo de herramientas como el \textit{dynamic reconfigure} de ROS han permitido realizar el ajuste de las ganancias de los controladores \textit{online}, lo que ha reducido considerablemente los tiempos necesarios para realizar estos ajustes.

Por otro lado, la decisión de separar la generación de trayectorias en dos partes, una a largo plazo y otra a largo plazo, ha permitido generar trayectorias de control óptimas en \textit{snap} con una alta reactividad frente a los cambios en las estimaciones de las puertas, manteniendo un bajo coste computacional en la generación de ambas trayectorias.

Finalmente, se han validado los desarrollos e implementaciones realizados en un entorno de simulación fotorrealista, consiguiendo completar el circuito de carreras completo de forma satisfactoria con una velocidad máxima de vuelo de 10,5 m/s en un tiempo inferior a los 23 segundos, lo que se situaría dentro de los mejores obtenidos el año pasado durante las clasificatorias virtuales del AlphaPilot2019.



\section{Trabajo futuro}

El trabajo realizado deja una arquitectura modular que permite su ampliación con nuevos módulos para adecuarla a la realización de distintas tareas de forma sencilla. Para una utilización del sistema en un caso real sería necesario implementar los módulos de estimación y percepción del entorno de una forma integral, empleando únicamente las medidas obtenidas por los sensores de la aeronave. 

Para mejorar el comportamiento del controlador sería conveniente emplear un controlador predictivo basado en el modelo (MPC), que permitiría reducir el error de seguimiento suavizando las acciones bruscas realizadas por el controlador cuando sufre cambios en la trayectoria de referencia. Junto con este controlador, se podrían emplear algoritmos de auto-identificación que permitan estimar \textit{online} el valor de los parámetros dinámicos de la aeronave, así como identificar posibles perturbaciones presentes en el entorno como podría ser el viento.

Finalmente, se podrían implementar métodos de optimización para calcular trayectorias que tengan en cuenta restricciones espaciales, lo que permitiría su empleo en entornos complicados con obstáculos alrededor. 












	\appendix
 
\chapter{Presupuesto y Planificación}

\section{Presupuesto}

El presupuesto del trabajo se puede separar en dos partes: recursos humanos y amortización de los equipos utilizados.

En cuanto a los recursos humanos empleados, se ha tenido una dedicación por parte del alumno de unas 400 horas, lo que se corresponde dentro del número de horas de dedicación esperadas en la realización de un Trabajo fin de Máster (12 ECTS). Un salario mensual de ayudante investigador a jornada completa en la universidad, es de unos 1250 euros, lo que se traduce en un coste de unos 8,33 euros la hora. El salario del tutor se ha extraído del portal de transparencia de la UPM. La dedicación del tutor ha sido de unas 30 horas de implicación en el trabajo.

\begin{figure}[htb!]
		\centering
		\begin{tabular}{|l|r|r|r|}
		\hline
		%\textbf{Recursos humanos} &Coste unitario [EUR] &Unidades&Total [EUR]\\
		
		\textbf{Recursos humanos} & Horas &Coste Horario [EUR]&Total [EUR]\\
		\hline
		
		Alumno & 400 & 8.33 &  3332 \\
		Tutor & 30& 33.72 & 1011.6 \\
		\hline
		\textbf{Total} & &  & \textbf{4343.6}\\
		\hline
		\end{tabular}\\
	
\end{figure}


En cuanto a la amortización del equipo, se ha empleado un ordenador para el desarrollo del software y para la realización de los experimentos. Se ha considerado una amortización lineal del 10 \% de la vida útil (4 años).

\begin{figure}[htb!]
	\centering
	\begin{tabular}{|l|r|r|r|}
		\hline
		\textbf{Equipo} & Precio &Coste Amortización(10\%)\\
		\hline
		Pc sobremesa & 1980 & 198\\
		\hline
		\textbf{Total} & & \textbf{198}\\
		\hline
\end{tabular}\\
\end{figure}

Con lo que el coste total del proyecto ha sido:

\begin{figure}[htb!]
	\centering
	\begin{tabular}{|l|r|r|r|}
		\hline
		\textbf{Concepto} &Total [EUR]\\
		\hline
		Recursos humanos & 4343.6\\
		Amortización del equipo & 198\\
		
		\hline
		\textbf{Total}   & \textbf{4541.6}\\
		\hline
	\end{tabular}\\
\end{figure}




\newpage
\section{Planificación}
La realización de este trabajo ha empleado un ritmo continuo de horas de trabajo desde su comienzo, a comienzos de abril. La dedicación media invertida en el desarrolo del trabajo ha sido de unas 35 horas semanales, durante un periodo de unos 3 meses, lo que da un total de unas 400 horas. El trabajo se ha realizado dentro del grupo de investigación CVAR (\textit{Computer Vision and Aerial Robotics}) del departamento de Electrónica Automática e Informática industrial de la Escuela Técnica Superior de Ingenieros Industriales (ETSII) perteneciente a la Universidad Politécnica de Madrid (UPM).

En cuanto a la distribución del trabajo en este tiempo, el trabajo comenzó a realizarse a finales de abril de 2020, durante el primer mes se realizó el curso sobre robótica aérea de UPenn, en la plataforma \textit{online} Coursera. La duración del curso se extendió hasta finales de abril. En mayo el trabajo se focalizó en el desarrollo de la arquitectura que se emplearía posteriormente, así como en el estudio del arte de los diferentes algoritmos de control y métodos de generación de trayectorias relevantes en el campo a estudiar. En junio y julio, se realizaron las implementaciones de los distintos algoritmos y se realizaron los primeros experimentos. Finalmente se ha redactado la memoria del proyecto en el mes de Septiembre. Se ha realizado un diagrama GANTT (\cref{gantt}) en el que se ha detallado más en profundidad la distribución temporal de las tareas. Asimismo, se ha esquematizado la organización del proyecto en un diagrama EDP 

\vspace{0.5cm}
\begin{figure}[htb!]
	\centering
	\includegraphics[width=0.85\textwidth]{imagenes/EDP}
	\caption{Diagrama EDP}
	\label{edp}
\end{figure}

\newpage

\textcolor{white}{aligment}
\begin{figure}[htb!]
	\centering
	\includegraphics[angle=90,width =0.45\textwidth]{imagenes/gantt}
	\caption{Diagrama Gantt}
	\label{gantt}
\end{figure}


	
%	\pagenumbering{arabic}
%	
	\newpage
	\listoffigures
		\nocite{*}
	\bibliographystyle{bibliografia/IEEEtran}
	\bibliography{bibliografia/IEEEabrv,bibliografia/workBibliography}
	
\end{document}
