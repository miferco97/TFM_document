\chapter{Generación de trayectorias}

En el apartado anterior se ha diseñado un controlador cuyo objetivo es seguir una trayectoria \tb{$x^{des}(t)$} minimizando el error de seguimiento. En este apartado se tratará sobre la forma en la que se generan estas trayectorias.\\

El objetivo es generar trayectorias a través del circuito de forma que la aeronave sea capaz de recorrerlas teniendo en cuenta restricciones de posición, velocidad y aceleración.
\tb{\Large control optimo antes ???} Se ha decidido usar trayectorias de tipo \textit{spline}, es decir, curvas diferenciables definidas en segmentos polinómicos. Cada trayectoria $r_d(t)$ está compuesta por $m$ segmentos de grado $q$, presentando la estructura

\begin{equation}
	r_d(t) = \left\{ 
	\begin{array}{ll}
		\sum_{i=1}^{q}a_{i1}t^{i} &,t_0\leq x \leq t_1 \\
		\sum_{i=1}^{q}a_{i2}t^{i} &,t_1\leq x \leq t_2 \\
		\qquad\vdots &\qquad\vdots \\
		\sum_{i=1}^{q}a_{im}t^{i} &,t_{m-1}\leq x \leq t_m \\
	\end{array}
	\right.
\end{equation}

siendo $a_{ij}\in \mathbb{R}$ los coeficientes de los polinomios. El grado $q$ de estos polinomios dependerá del índice a minimizar durante el transcurso de la trayectoria, como se explicará en el apartado \ref{trajectoriasoptimas:cap}.


\section{Trayectorias óptimas}\label{trajectoriasoptimas:cap}
El objetivo del control óptimo es encontrar la función $x^*(t)$ que minimiza la expresión
\begin{equation}
	x^*(t) =  \underset{x(t)}{argmin}\int_{0}^{T}\mathcal{L}\left(x^{(n)},x^{(n-1)},...,\dot{x},x,t\right)\, dt
\end{equation}
siendo $\mathcal{L}$ el índice que se debe optimizar.


